\section{Dualita}

\subsection{Pomocný důkaz vlastnosti infima}\label{infPomoc}
\[
    \inf_{(x, y)^T \in M \times N} (h_1(x) + h_2(y)) = \inf_{x\in M}h_1(x) + \inf_{y\in N}h_2(x)
\]

Důkaz.\\
\enquote{$\geq$}:
\begin{align*}
    \inf_{x \in M} h_1(x) \leq h_1(t) \quad \forall t \in M \\
    \inf_{y \in N} h_2(y) \leq h_2(t) \quad \forall t \in N
\end{align*}
\[
    \implies \inf_{x\in M} h_1(x) + \inf_{y\in N}h_2(y) \leq \inf_{(x,y)^T \in M \times N}(h_1(x) + h_2(y)) \qed
\]
\enquote{$\leq$}:
\[
    \inf_{(x,y)^T \in M \times N}(h_1(x) + h_2(y)) \leq h_1(t) + h_2(s) \quad \forall 
    \begin{bmatrix}
        t \\
        s    
    \end{bmatrix} \in M \times N
\]
\[
    \text{což lze upravit: } -h_2(s) + \inf_{(x, y)^T \in M \times N}(h_1(x) + h_2(y)) \leq h_1(t) \quad \forall t\in M, 
    \forall s\in N.
\]
\[
    \implies -h_2(s) + \inf_{(x, y)^T \in M \times N}(h_1(x) + h_2(y)) \leq \inf_{x\in M} h_1(x) \quad \forall s\in N.
\]
A to samé lze ukázat i pro $h_2$:
\[
    -h_1(t) + \inf_{(x, y)^T \in M \times N}(h_1(x) + h_2(y)) \leq h_2(s) \quad \forall t\in M, \forall s\in N.
\]
\[
    \implies -h_1(t) + \inf_{(x, y)^T \in M \times N}(h_1(x) + h_2(y)) \leq \inf_{y\in N} h_2(y) \quad \forall t\in M.
\]
Teď sečtěme tyto dvě nerovnice:
\[
    -h_1(t) - h_2(s) + 2 \inf_{(x, y)^T \in M \times N}(h_1(x) + h_2(y)) \leq \inf_{x\in M} h_1(x) + \inf_{y\in N} h_2(y)
    \quad \forall t\in M, \forall s\in N.
\]
Tedy stačí dokázat, že
\begin{align*}
    \inf_{(x,y)^T \in M \times N}(h_1(x) + h_2(y)) &\leq -h_1(t) - h_2(s) + 2 \inf_{(x, y)^T \in M \times N}(h_1(x) + h_2(y)) 
    \quad \forall t\in M, \forall s\in N.\\
    0 &\leq -h_1(t) - h_2(s) + \inf_{(x, y)^T \in M \times N}(h_1(x) + h_2(y)) \quad \forall t\in M, \forall s\in N.\\
    h_1(t) + h_2(s) &\leq \inf_{(x, y)^T \in M \times N}(h_1(x) + h_2(y)) \quad \forall t\in M, \forall s\in N.
\end{align*}

\subsection{Dualita - motivační příklad}
Je dána úloha
\Opt{min}{}{2x+3y}{
    1-x-y \: &\leq 0,\\
    x,y \: &\in [0,2].
}

Označme $f(x,y) = 2x + 3y$,
$
    M = \bc{
    \begin{bmatrix}
        x \\
        y
    \end{bmatrix} \in [0,2]^2 \, \middle| \, 1 - x - y \leq 0}
$
a $\hat f = \min_{x\in M} f(x)$.

Odhadněme min funkce ze spoda. 

Pro $(x, y)^T \in M$: \[f(x, y) \geq f(x,y) + g_1(x,y) = 2x + 3y + (1-x-y) = x+2y+1 \geq 1.\]
A protože $\hat f = \min f(x)$, nutně musí platit $\hat f \geq 1$.

Zkusme teď jiný odhad.
\[f(x, y) \geq f(x,y) + 2g_1(x,y) = 2x + 3y + 2(1-x-y) = 2y + 1 \geq 2.\]
Nalezli jsme lepší odhad: $\hat f \geq 2$. Jak tedy správně určit \enquote{nejlepší} možný dolní odhad $\hat f$?

Definujme si
\begin{align*}
    L(x, y, \mu) &= 2x + 3y + \mu(1-x-y), \\
    \varphi(\mu) &= \min_{(x, y)^T \in [0,2]^2}L(x, y, \mu).
\end{align*}
Pro každé $\mu \geq 0$ pak platí:
\[
    \varphi(\mu) = \min_{(x, y)^T \in \Omega}L(x, y, \mu) \leq \min_{(x, y)^T \in M}L(x, y, \mu) \leq \hat f
\]
\enquote{Optimální} dolní odhad $\hat f$ pomocí $\varphi$ vede na úlohu
\Opt{max}{}{\varphi(\mu)}{
    \mu \geq 0.
}
Kde
\begin{align*}
    \varphi(\mu) &= \min_{(x, y)^T \in [0,2]^T} \left[(2-\mu)x + (3-\mu)y + \mu\right] \\
    &= \mu + \min_{x \in [0,2]}(2-\mu)x + \min_{y \in [0,2]}(3-\mu)y \\
    &= 
    \begin{cases}
        \mu & \mu < 2 \\
        \mu + 4 - 2\mu & \mu \in [2, 3) \\
        10 - 3\mu & \mu \in [3, \infty)
    \end{cases}
\end{align*}
Tu budeme nazývat duální úlohou.

Hodnota $\max \varphi(\mu)$ na $[0, +\infty)$ je $\hat \varphi \implies \hat \varphi \leq \hat f$.

\subsection{Tvrzení o konkávnosti duální úlohy}
Jestliže $D_\varphi \not= \emptyset$, pak $\varphi$ je konkávní.

Důkaz.\\
Mějme $\mu, \nu \in D_\varphi$, $\lambda \in [0,1]$.
\begin{align*}
    \varphi(\lambda \mu + (1-\lambda)\nu) &\overset{\text{?}}{=} \inf_{x\in \Omega} \overbrace{f(x)}^{\lambda f(x) + 
    (1-\lambda)f(x)} + \overbrace{\langle g(x), \lambda \mu + (1-\lambda)\nu\rangle}^{\lambda \langle g(x), \mu\rangle + 
    (1-\lambda)\langle g(x), \nu\rangle} \\
    &= \inf_{x\in \Omega} \lambda(f(x) + \langle g(x), \mu\rangle) + (1-\lambda)(f(x)+\langle g(x), \nu\rangle) \\
    &\hspace*{-1.1em}\underset{\text{\hyperref[infPomoc]{infima}}}{\overset{\text{\hyperref[infPomoc]{vlastnost}}}{\geq}} 
    \lambda \inf_{x\in \Omega} (f(x) + \langle g(x), \mu\rangle) + (1 - \lambda) \inf_{x\in \Omega} (f(x) + \langle g(x), 
    \nu\rangle) \\
    &= \lambda \varphi(\mu) + (1-\lambda) \varphi(\nu) > - \infty \implies \lambda \mu + (1-\lambda)\nu \in D_\varphi. \qed
\end{align*}

\subsection{Věta o slabé dualitě}\label{slabDual}
\begin{enumerate}[(a)]
    \item Pro každé $x \in M$ a $\mu \in N$ je $\varphi(\mu) \leq f(x)$.
    \item $\hat \varphi \leq \hat f$.
\end{enumerate}
(a) Důkaz.\\
Víme: $L(x, \mu) \leq f(x) \quad \forall x \in M, \forall \mu \geq 0$.
\[
    \varphi(\mu) = \inf_{y \in \Omega} L(y, \mu) \leq \inf_{y \in M} L(y, \mu) \leq L(x, \mu) \leq f(x) \quad 
    \forall x \in M, \forall \mu \in N. \qed
\]
(b) Důkaz.\\
Z (a) máme $\overbrace{\sup_{\mu \in N} \varphi(\mu)}^{\hat \varphi} \leq f(x) \quad \forall x \in M$.
\[
    \implies \hat \varphi \leq \inf_{x \in M}f(x) = \hat f. \qed
\]

\subsection{Důsledek věty o slabé dualitě}
\begin{enumerate}[(a)]
    \item Jestliže existují $\hat x \in M$ a $\hat \mu \in N$ splňující $\varphi(\hat \mu) = f(\hat x)$, pak
    \[ \hat \mu \in \argmax_{\mu \in N} \varphi(\mu) \quad \text{ a } \quad \hat x \in \argmin_{x \in M}f(x). \]
    \item Je-li $\hat f = - \infty$, pak $N = \emptyset$.
    \item Je-li $\hat \varphi = + \infty$, pak $M = \emptyset$.
\end{enumerate}
Důkaz (a).\\
Z \hyperref[slabDual]{věty o slabé dualitě} platí:
\[
    \varphi(\mu) \leq f(\hat x) \overset{\text{předpoklad}}{=} \varphi(\hat \mu) \quad \forall \mu \in N
    \iff \hat \mu \in \argmax_{\mu \in N} \varphi(\mu).
\]
Analogicky:
\[
    f(\hat x) \overset{\text{předpoklad}}{=} \varphi(\hat \mu) \leq f(x) \quad \forall x \in M
    \iff \hat x \in \argmin_{x \in M} f(x). \qed
\]
Důkaz (b).\\
Sporem. Ať $N \not= \emptyset$. Volme $\mu \in N$.
\[
    \text{Pak } \underbrace{\varphi(\mu)}_{\in \R} \leq \hat \varphi \leq \hat f = - \infty \dots \text{ spor.} \qed
\]
Důkaz (C).\\
Sporem. Ať $M \not= \emptyset$. Volme $x \in M, \mu \in N$.
\[
    \text{Pak } \varphi(\mu) \leq \hat \varphi = + \infty \leq \underbrace{f(x)}_{\in \R} \dots \text{ spor.} \qed
\]

\subsection{Ukázkový příklad na slabou dualitu}
Je dána úloha
\Opt{min}{}{-x^2}{
    2x - 1 \: &\leq 0, \\
    x \: &\in [0,1].
}
Tedy:
\[
    L(x, \mu) = -x^2 + \mu(2x-1) = (-x^2 + 2x\mu) - \mu
\]
\[
    \varphi(\mu) = \left[\min_{x \in [0,1]}(-x^2 + 2x\mu)\right] - \mu
\]
Pozorování. Minimalisovaná funkce je (ryze) konkávní. Nemůže tedy v žádném vnitřním bodě nabývat minima. Dosazení 
krajních bodů intervalu:
\[
    \varphi(\mu) = \min \bc{0, 2\mu -1} - \mu = 
    \begin{cases}
        \mu -1 & \text{pro } \mu < \frac{1}{2}, \\
        -\mu & \text{pro } \mu \geq \frac{1}{2}.
    \end{cases}
\]
\begin{multicols}{2}
    \begin{tikzpicture}
        \begin{axis}[
            axis lines=middle,
            xlabel={\(\mu\)},
            ylabel={\(\varphi(\mu)\)},
            xmin=-1.9, xmax=1.9,
            ymin=-1.9, ymax=1.9,
            samples=100,
            domain=-2:2,
            legend pos=north east,
        ]
            \addplot [blue, thick, domain=-2:0.5] {x - 1};
            \addplot [blue, thick, domain=0.5:2] {-x};
            % \addlegendentry{\(\varphi(\mu)\)}
    
            \addplot [only marks, mark=*] coordinates {(0.5,-0.5)};
            
            \draw [dotted] (axis cs:0.5,-0.5) -- (axis cs:0.5,0);
            \draw [dotted] (axis cs:0.5,-0.5) -- (axis cs:0,-0.5);
            
            \addplot [only marks, mark=*] coordinates {(0.5,0)};
            \node [above right] at (axis cs:0.5,0) {\(\frac{1}{2}\)};
            
            \addplot [only marks, mark=*] coordinates {(0,-0.5)};
            \node [above left] at (axis cs:0,-0.5) {\(-\frac{1}{2}\)};
        \end{axis}
    \end{tikzpicture}

\columnbreak

    Z grafu vyčteme: $\hat \varphi = - \frac{1}{2}$. A to samé uděláme pro $f$, kde výsledek bude $\hat f = - \frac{1}{4}$.

    Tedy $\hat \varphi < \hat f$.
\end{multicols}

\subsection{Věta o silné dualitě}\label{silDual}
Nechť $\hat f < \infty$ a cílová funkce $f : \R^n \rightarrow \R$ je konvexní. Předpokládejme, že platí alespoň jedna z 
následujících podmínek:
\begin{enumerate}[(a)]
    \item Komponenty $g_1, \dots, g_k$ zobrazení $g$ splňují \hyperref[slaterPodm]{Slaterovu podmínku} regularity.
    \item Zobrazení $g$ je afinní a $\Omega$ je konvexní polyedrická množina.
\end{enumerate}
Potom $\hat f = \hat \varphi$. Je-li navíc $\hat f \in \R$, pak existuje řešení úlohy ($D$).

Důkaz vynecháme.