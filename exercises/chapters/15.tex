\section{Řešená vzorová písemka}

\subsection{Konvexní funkce}
Je dána funkce
\[
    f(x_1, x_2, x_3) = e^{x_3} + x_1^2 - 2 \alpha x_1x_2 + x_2^4,
\]
kde $\alpha \in \R$ je parametr.
\begin{enumerate}[(a)]
    \item Pro jaké hodnoty parametru $\alpha$ je $f$ konvexní?
    \item Ukažte, že pro $\alpha = 0$ je množina
    \[
        M = \bc{(x_1, x_2, x_3)^T \in \R^3 \mid f(x_1, x_2, x_3) \leq 1, x_1 \geq 0, x_2 \geq 0}
    \]
    konvexní.
\end{enumerate}

(a) Určíme definitnost Hessiánu funkce. Pokud bude positivně (semi)definitní, funkce bude (ne)ryze konvexní.
\begin{align*}
    f^\prime (x_1, x_2, x_3) &= \left(2x_1 - 2\alpha x_2, -2\alpha x_1 + 4 x_2^3, e^{x_3}\right) \\
    f^{\prime \prime}(x_1, x_2, x_3) &= 
    \begin{bmatrix}
        \phantom{-}2 & -2\alpha & 0 \\
        -2\alpha & \phantom{-}12x_2^2 & 0 \\
        \phantom{-}0 & \phantom{-}0 & e^{x_3}    
    \end{bmatrix}
\end{align*}
Použijme Sylvesterova kritéria k určení definitnosti.\\
$\begin{vmatrix}2\end{vmatrix} = 2 \geq 0$\\
$\begin{vmatrix}
    \phantom{-}2 & -2\alpha \\
    -2\alpha & \phantom{-}12x_2^2
\end{vmatrix} = 24x_2^2 - 4\alpha^2 \geq 0 \iff 6x_2^2 \geq \alpha^2$ \dots což lze zajistit jen tehdy, když $\alpha=0$,
protože nemůžeme omezit hodnoty $x_2$.

A proto funkce $f$ bude konvexní právě tehdy, když $\alpha = 0$.

(b) Z předchozího bodu víme, že funkce $f$ je konvexní právě tehdy, když $\alpha = 0$. \\ Podmínka 
$f(x_1, x_2, x_3) \leq 1$ je pouhá dolní úrovňová množina. A ta je určitě konvexní, protože je původní funkce $f$ 
konvexní za těchto podmínek. 

Zbylé podmínky $x_1 \geq 0$ a $x_2 \geq 0$ jsou rozhodně konvexní. 

A protože průnik zachovává konvexitu, tak průnik těchto 3 podmínek je stále konvexní. Množina $M$ je tedy konvenxí.

\subsection{Metoda nejmenších čtverců}
Nechť
\[
    A = 
    \begin{bmatrix}
    -1 & \phantom{-}0 \\
    \phantom{-}2 & -2 \\
    \phantom{-}0 & \phantom{-}1    
    \end{bmatrix} \: \text{ a } \: b =
    \begin{bmatrix}
        1 \\ 2  \\ 0
    \end{bmatrix}.
\]
Nalezněte všechny body minima funkce $f(x) = \|Ax - b\|^2$ na $\R^2$.

Pokusíme se použít $(A^T A)^{-1} A^T A x = (A^T A)^{-1} A^T b$.
\begin{align*}
    A^T A = 
    \begin{bmatrix}
        -1 & \phantom{-}2 & 0 \\
        \phantom{-}0 & -2 & 1    
    \end{bmatrix}
    \begin{bmatrix}
        -1 & \phantom{-}0 \\
        \phantom{-}2 & -2 \\
        \phantom{-}0 & \phantom{-}1   
    \end{bmatrix} = 
    \begin{bmatrix}
        5 & -4 \\
        -4 & 5    
    \end{bmatrix} \\
    \det(A^T A) = 
    \begin{vmatrix}
        5 & -4 \\
        -4 & 5  
    \end{vmatrix} = 25 - 16 = 9 > 0 \dots \text{ existuje inverze.}
\end{align*}
\[
    (A^T A)^{-1} = \frac{1}{9}
    \begin{bmatrix}
        5 & 4 \\
        4 & 5    
    \end{bmatrix}
\]
\begin{align*}
    \hat x = (A^T A)^{-1} A^T b = \frac{1}{9}
    \begin{bmatrix}
        5 & 4 \\
        4 & 5    
    \end{bmatrix}
    \begin{bmatrix}
        -1 & \phantom{-}2 & 0 \\
        \phantom{-}0 & -2 & 1    
    \end{bmatrix}
    \begin{bmatrix}
        1 \\ 2  \\ 0
    \end{bmatrix} = \frac{1}{9}
    \begin{bmatrix}
        5 & 4 \\
        4 & 5    
    \end{bmatrix}
    \begin{bmatrix}
        \phantom{-}3 \\ -4 
    \end{bmatrix} = \frac{1}{9}
    \begin{bmatrix}
        -1 \\ -8
    \end{bmatrix}
\end{align*}

\subsection{KKT podmínky}
Je dána úloha
\Opt{min}{}{-2x_1 + x_2}{
    x_1 - x_2 \: &\leq 0, \\
    x_1^2 + x_2^2 \: &\leq 8.
}
\begin{enumerate}[(a)]
    \item Napište KKT podmínky pro tuto úlohu.
    \item Ověřte, že KKT podmínky jsou splněny v bodě $\begin{bmatrix}2 \\ 2\end{bmatrix}$.
    \item Využitím KKT podmínek zdůvodněte, proč $\begin{bmatrix}2 \\ 2\end{bmatrix}$ je řešení zadané úlohy.
\end{enumerate}

(a)
\begin{align*}
    f(x_1, x_2) &= -2x_1 + x_2 \\
    g_1(x_1, x_2) &= x_1 - x_2 \\
    g_2(x_1, x_2) &= x_1^2 + x_2^2 - 8
\end{align*}
\begin{align*}
    \nabla f(x_1, x_2) &= (-2, 1) \\
    \nabla g_1(x_1, x_2) &= (1, -1) \\
    \nabla g_2(x_1, x_2) &= (2x_1, 2x_2)
\end{align*}

KKT podmínky jsou:
\begin{align*}
    -2 + \mu_1 + \mu_2 \cdot 2x_1 &= 0\\
    1 - \mu_1 + \mu_2 \cdot 2x_2 &= 0\\
    \mu_1(x_1 - x_2) &= 0\\
    \mu_2(x_1^2 + x_2^2 - 8) &= 0\\
    \mu_1, \mu_2 &\geq 0
\end{align*}
(b)
Dosadíme do KKT podmínek a ověříme, že všechny podmínky jsou splněny.
\begin{align*}
    -2 + \mu_1 + 4\mu_2 &= 0 \quad (\text{I}) \\
    1 - \mu_1 + 4\mu_2&= 0 \quad (\text{II})\\
    \mu_1(0) &= 0 \quad ✓\\
    \mu_2(0) &= 0 \quad ✓\\
    \mu_1, \mu_2 &\geq 0
\end{align*}
Po odečtení (I)-(II) nám vyjde $\mu_1 = \frac{3}{2}$ a $\mu_2 = \frac{1}{8}$, což je v souladu s podmínkami.

(c)
Afinní podmínka regularity není splněna, ověříme Slaterovu.

$g_1$ je afinní $\implies$ konvexní funkce. U $g_2$ musíme ověřit definitnost Hessiánu.
\[
    \nabla^2 g_2(x_1, x_2) = 
    \begin{bmatrix}
        2 & 0 \\
        0 & 2
    \end{bmatrix}
\]
Matice je symetrická, můžeme tedy ověřovat definitnost. Využijeme Sylvesterova pravidla.

$\begin{vmatrix}
    2
\end{vmatrix} = 2 > 0$, $
\begin{vmatrix}
    2 & 0 \\
    0 & 2    
\end{vmatrix} = 4 > 0$. $g_2$ je (ryze) konvexní. Když se nám podaří nalézt $x \in \Omega$ takové, že $g_i(x) < 0$, pak 
bude Slaterova podmínka splněna. Zvolme $x = \begin{bmatrix} 0 \\ 1 \end{bmatrix}$, pak očividně $g_i(x) < 0$, Slaterova
podmínka regularity je splněna. Dále $f(x_1, x_2)$ je afinní, tedy konvexní. $\implies$ Postačující KKT podmínky nám 
zaručují, že když bod $\begin{bmatrix}2 \\ 2\end{bmatrix}$ je KKT bodem, pak je bodem minima funkce $f$.

\subsection{Smíšené rozšíření maticové hry}
Nechť $\Gamma(A)$ je smíšené rozšíření maticové hry s maticí hry
\[
    A = 
    \begin{bmatrix}
        \phantom{-}2 & 0 & 3 \\
        -1 & 1 & 2 \\
        \phantom{-}1 & 3 & 0
    \end{bmatrix}.
\]
Pomocí úlohy lineárního programování nalezněte cenu hry $\Gamma(A)$ a optimální strategii prvního hráče.