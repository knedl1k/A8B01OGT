\section{Úvod do strategických her}

\subsection{Příklad Vězňovo dilemma}
Hra je daná tabulkou:
\[
\begin{array}{|c| |c|c|}
    \hline
    & P & Z \\ \hline \hline
    P & -5;-5 & 0;-10 \\ \hline
    Z & -10;0 & -1;-1 \\ \hline
\end{array}
\]
$N = \bc{1,2}$.\\
$S_1 = S_2 = \bc{P, Z}$.

Funkce úžitku:
\begin{align*}
    u_1(P, P) &= -5\phantom{0} = u_2(P, P) \\
    u_1(P, Z) &= \phantom{-}0\phantom{0} = u_2(Z, P) \\
    u_1(Z, P) &= -10 = u_2(P, Z) \\
    u_1(Z, Z) &= -1\phantom{0} = u_2(Z, Z)
\end{align*}

\subsection{Příklad Panna nebo orel}
Hra je daná tabulkou:
\[
\begin{array}{|c| |c|c|}
    \hline
    & P & O \\ \hline \hline
    P & 10;-10 & -10;10 \\ \hline
    O & -10;10 & 10;-10 \\ \hline
\end{array}
\]
První hráč dostane body, pokud se budou oba hráči shodovat. Druhý hráč dostane body, pokud budou odlišné.

\subsection{Příklad Manželský spor}
Hra je daná tabulkou:
\[
\begin{array}{|c| |c|c|}
    \hline
    & D & H \\ \hline \hline
    D & 2;3 & -1;-1 \\ \hline
    H & 0;0 & 3;2 \\ \hline
\end{array}
\]
Hokej a Divadlo. Čísla jsou radosti.

\subsection{Příklad Kámen-nůžky-papír}
Hra je daná tabulkou:
\[
\begin{array}{|c| |c|c|c|}
    \hline
    & K & N & P \\ \hline \hline
    K & 0;0 & 1;-1 & -1;1 \\ \hline
    N & -1;1 & 0;0 & 1;-1 \\ \hline
    P & 1;-1 & -1;1 & 0;0 \\ \hline
\end{array}
\]
\newpage
\subsection{Nashovo equilibrium}\label{nash}
Nechť $G = (N, (S_i)^n_{i=1}, (u_i)^n_{i=1})$ je strategická hra. Strategický profil $\hat \sigma = 
(\hat \sigma_1, \dots, \hat \sigma_n) \in S$ se nazve Nashovo equilibrium hry $G$, jestliže pro každé $i \in N$ a každé
$\sigma_i \in S_i$ je 
\[
    u_i(\hat \sigma_1, \dots, \hat \sigma_n) \geq u_i (\hat\sigma_1, \dots, \hat\sigma_{i-1}, \sigma_i, 
    \hat \sigma_{i+1}, \dots, \hat \sigma_n)
\]
Nashovo equilibrium nám říká, že hráč, pouze změnou své strategie, si nemůže polepšit. Nevede k 
\enquote{maximalisaci zisku}, ale k rovnováze.\\
Zároveň N.e. nemusí být určeno jednoznačně, dokonce ani nemusí existovat.\\
Speciálně pokud $N = \bc{1,2}$.
\begin{itemize}
    \item $u_1(\sigma_1, \hat \sigma_2) \leq u_1(\hat \sigma_1, \hat \sigma_2)$ $\forall \sigma_1 \in S_1$,
    \item $u_2(\hat \sigma_1, \sigma_2) \leq u_2(\hat \sigma_1, \hat \sigma_2)$ $\forall \sigma_2 \in S_2$.
\end{itemize}

\subsection{Vězňovo dilemma a Nashovo equilibrium}
\[
\begin{array}{|c| |c|c|}
    \hline
    & P & Z \\ \hline \hline
    P & -5;-5 & 0;-10 \\ \hline
    Z & -10;0 & -1;-1 \\ \hline
\end{array}
\]
\begin{enumerate}[(a)]
    \item Strategický profil $(P, P)$: 
    \begin{align*}
        u_1(Z, P) \leq u_1(P, P) \checkmark \\
        u_1(P, P) \leq u_1(P, P) \checkmark \\
        u_2(P, Z) \leq u_2(P, P) \checkmark \\
        u_2(P, P) \leq u_2(P, P) \checkmark
    \end{align*}
    $\implies (P, P)$ je N. e.
    \item Strategický profil $(P, Z)$: Zde není N. e., neboť 
    \[
        u_2(P, Z) < u_2 (P, P).
    \]
    \item Strategický profil $(P, Z)$: Zde není N. e., neboť 
    \[
        u_1(Z, P) < u_1 (P, P).
    \]
    \item Strategický profil $(Z, Z)$: Zde není N. e., neboť 
    \[
        u_1(Z, Z) < u_1 (P, Z).
    \]
\end{enumerate}

\subsection{Panna nebo orel a Nashovo equilibrium}
\[
\begin{array}{|c| |c|c|}
    \hline
    & P & O \\ \hline \hline
    P & 10;-10 & -10;10 \\ \hline
    O & -10;10 & 10;-10 \\ \hline
\end{array}
\]
Zde N.e. neexistuje.

\subsection{Manželský spor a Nashovo equilibrium}
\[
\begin{array}{|c| |c|c|}
    \hline
    & D & H \\ \hline \hline
    D & 2;3 & -1;-1 \\ \hline
    H & 0;0 & 3;2 \\ \hline
\end{array}
\]
Strategické profily $(D, D)$ a $(H,H)$ jsou jediná N.e. v této hře.
\newpage
\subsection{Tvrzení o Nashově equilibriu}\label{tvrNash}
Nechť $G = (N, (S_i)^n_{i=1}, (u_i)^n_{i=1})$ je strategická hra a $\hat \sigma \in S$. Pak následující tvrzení jsou 
ekvivalentní:
\begin{enumerate}[(a)]
    \item $\hat \sigma$ je Nashovo equilibrium.
    \item Pro každé $i \in N$ je
    \[
        \hat \sigma_i \in \argmax_{\sigma_i \in S_i} u_i (\hat \sigma_i, \dots, \hat \sigma_{i-1}, \sigma_i, 
        \hat \sigma_{i+1}, \dots, \hat \sigma_n).
    \]
\end{enumerate}
Důkaz plyne přímo z definice Nashova equilibria.

\subsection{Příklad Cournotův model oligopolu a Nashovo equilibrium}
Ať
\begin{itemize}
    \item $N = \bc{1,2}$ (tj. uvažujeme model duopolu) a $S_1 = S_2 = [0, \infty)$.
    \item $C_1(q_1) = cq_1$ a $C_2(q_2) = cq_2$, kde $c>0$.
    \item $P(q_1 + q_2) = a- b(q_1 + q_2)$, kde $a>c$ a $b>0$.
    \item $u_1(q_1, q_2) = [a-b(q_1 + q_2)]q_1 - cq_1$ a \\
    $u_2(q_1, q_2) = [a-b(q_1 + q_2)]q_2 - cq_2$.
\end{itemize}
\begin{align}
    \frac{\partial u_1}{\partial q_1} = 0 \dots -bq_1 + a-b(q_1 + q_2) -c = 0 \rightarrow -2bq_1 = -a + c + bq_2 \\
    \frac{\partial u_2}{\partial q_2} = 0 \dots -bq_2 + a-b(q_1 + q_2) -c = 0 \rightarrow -2bq_2 = -a + c + bq_1
\end{align}
Dosadíme do (5):
\[
    -2bq_1 = -a + c + \frac{a-c}{2} - \frac{bq_1}{2} \implies \left(\frac{b}{2} - 2b\right)q_1 = -\frac{a-c}{2}
\]
\[
    \implies q_1 = \frac{a-c}{3b} > 0
\]
Z (6):
\[
    q_2 = \frac{a-c}{2b} - \frac{1}{2}\left(\frac{a-c}{3b}\right) = \frac{3 (a-c) - (a-c)}{6b} = \frac{a-c}{3b} > 0
\]
Díky \hyperref[tvrNash]{tvrzení o Nashově equilibriu} jsme našli dvojici, která je Nashovým equilibriem: 
$\left(\frac{a-c}{3b}, \frac{a-c}{3b}\right)$.
\newpage
\subsection{Hra dvou hráčů s nulovým součtem}
Hra dvou hráčů s nulovým součtem je strategická hra
\[
    G = (\bc{1,2}, (S_1, S_2), (u_1, u_2))
\]
taková, že pro každé $(\sigma_1, \sigma_2) \in S_1 \times S_2$ je
\[
    u_1(\sigma_1, \sigma_2) + u_2(\sigma_1, \sigma_2) = 0.
\]
\begin{itemize}
    \item Hráči mají zcela opačné zájmy.
    \item Stačí zadat jen jednu funkci užitku, neboť $u_1 = -u_2$.
    \item Je zbytečné uvádět množinu $\bc{1,2}$ všech hráčů.
    \item Zjednodušené značení hry dvou hráčů:
    \[G = (S_1, S_2, u) \text{, kde } u = u_1.\]
\end{itemize}

\subsection{Definice ceny hry}\label{cenaHry}
Ať $G = (S_1, S_2, u)$ je hra dvou hráčů s nulovým součtem.
\begin{enumerate}[(a)]
    \item Dolní cena hry $G$ je číslo
    \[\underline{v} \coloneqq \sup_{\sigma \in S_1} \inf_{\tau \in S_2} u(\sigma, \tau).\]
    \item Horní cena hry $G$ je číslo
    \[\overline{v} \coloneqq \inf_{\tau \in S_2} \sup_{\sigma \in S_1} u(\sigma, \tau).\]
    \item Řekněme, že $v \in \R$ je \textbf{cena hry} $G$, jestliže $v = \underline{v} = \overline{v}$.
\end{enumerate}
Pozorování.
\begin{itemize}
    \item První hráč nemůže \enquote{získat} méně, než $\underline{v}$.
    \item Druhý hráč nemůže \enquote{prohrát} více, než $\overline{v}$.
    \item Platí $\underline{v} \leq \overline{v}$, neboť:
    \[
        \inf_{\tau \in S_2} u(\sigma, \tau) \leq u(\sigma, \tilde{\tau}) \quad \forall \sigma \in S_1, 
        \forall \tilde{\tau} \in S_2
    \]
    Aplikujme supremum:
    \[  
        \sup_{\sigma \in S_1} \inf_{\tau \in S_2} u(\sigma, \tau) \leq \sup_{\sigma \in S_1} u(\sigma, \tilde{\tau}) 
        \quad \forall \tilde{\tau} \in S_2
    \]
    Levá strana je dolní odhad pravé. A infímum pravé je největší dolní mez.
    \[
        \underbrace{\sup_{\sigma \in S_1} \inf_{\tau \in S_2} u(\sigma, \tau)}_{\underline{v}} \leq 
        \underbrace{\inf_{\tau \in S_2} \sup_{\sigma \in S_1} u(\sigma, \tilde{\tau})}_{\overline{v}}
    \]
\end{itemize}

\subsection{Definice optimální strategie}\label{optimalStrat}
Ať $G = (S_1, S_2, u)$ je hra dvou hráčů s nulovým součtem a $v$ je její cena.
Řekněme, že
\begin{enumerate}[(a)]
    \item $\hat \sigma \in S_1$ je optimální strategie \textbf{prvního} hráče, jestliže
    \[v = \inf_{\tau \in S_2} u(\hat \sigma, \tau),\]
    \item $\hat \tau \in S_2$ je optimální strategie \textbf{druhého} hráče, jestliže
    \[v = \sup_{\sigma \in S_1} u(\sigma, \hat \tau).\]
\end{enumerate}

\subsection{Příklad na optimální strategii}
Hra $G$ je dána tabulkou:
\[
\begin{array}{|c| |c|c|}
    \hline
    & C & D \\ \hline \hline
    A & 1;-1 & 2;-2 \\ \hline
    B & 3;-3 & 4;-4 \\ \hline
\end{array}
\]
A protože $G$ je hra dvou hráčů s nulovým součtem, stačí zadat tabulku:
\[
\begin{array}{|c| |c|c|}
    \hline
    & C & D \\ \hline \hline
    A & 1 & 2 \\ \hline
    B & 3 & 4 \\ \hline
\end{array}
\]
Tedy $G = (S_1, S_2, u)$, kde $S_1 = \bc{A, B}$, $S_2 = \bc{C, D}$ a
\begin{align*}
    u(A,C) = 1, \\
    u(A,D) = 2, \\
    u(B,C) = 3, \\
    u(B,D) = 4.
\end{align*}
Určeme dolní cenu hry $G$:
\begin{align*}
    \inf_{\tau \in S_2} u(A, \tau) = 1, \\
    \inf_{\tau \in S_2} u(B, \tau) = 3. \\
\end{align*}
\[
    \underline{v} = \sup_{\sigma \in S_1} \inf_{\tau \in S_2} u(\sigma, \tau) = 3
\]
Obdobně horní cena hry $G$:
\[
    \overline{v} = \inf_{\tau \in S_2} \sup_{\sigma \in S_1} u(\sigma, \tau) = 3
\]
A proto je cena hry $v = 3$. Optimální strategie prvního hráče je pouze $B$. Optimální strategie druhého hráče je pouze 
$C$.
Shodou náhod je $(B,C)$ \hyperref[nash]{Nashovým equilibriem}.

\subsection{Optimální strategie Panna nebo orel}
$G$ je hra dvou hráčů s nulovým součtem, a proto stačí zadat tabulku:
\[
\begin{array}{|c| |c|c|}
    \hline
    & P & O \\ \hline \hline
    P & \phantom{-}10 & -10 \\ \hline
    O & -10 & \phantom{-}10 \\ \hline
\end{array}
\]
Určeme dolní cenu hry $G$:
\[
    \underline{v} = \sup_{\sigma \in S_1} \inf_{\tau \in S_2} u(\sigma, \tau) = -10
\]
Obdobně horní cena hry $G$:
\[
    \overline{v} = \inf_{\tau \in S_2} \sup_{\sigma \in S_1} u(\sigma, \tau) = 10
\]
Optimální strategie pro prvního i druhého hráče neexistuje, protože horní a dolní cena hry jsou rozdílné.

\subsection{Optimální strategie pouze pro jednoho hráče}
Uvažme hru $G=(S_1, S_2, u)$ dvou hráčů s nulovým součtem, kde $S_1 = S_2 = (0,1)$ a $u(\sigma, \tau) = \sigma \tau$.

Určeme dolní cenu hry $G$:
\[
    \underline{v} = \sup_{\sigma \in S_1} \inf_{\tau \in S_2} \sigma \tau = \sup_{\sigma \in S_1} 0 = 0
\]
Horní cena hry $G$:
\[
    \overline{v} = \inf_{\tau \in S_2} \sup_{\sigma \in S_1} \sigma \tau = \inf_{\tau \in S_2} \tau = 0
\]
A proto je cena hry $v = 0$. Optimální strategie prvního hráče je každá strategie z $S_1$. Optimální strategie druhého 
hráče neexistuje.

\subsection{Tvrzení o existenci optimální strategie}
Ať $G = (S_1, S_2, u)$ je hra dvou hráčů s nulovým součtem taková, že $S_1$ a $S_2$ jsou konečné. Jestliže existuje cena
hry $G$, pak nutně existuje optimální strategie prvního a také druhého hráče.

Důkaz.

Díky předpokladu, že $S_1$ a $S_2$ jsou konečné množiny, můžeme při výpočtech dolní, respektive horní, ceny hry nahradit
$\sup$ za $\max$, respetive $\inf$ za $\min$. A protože budeme hledat $\max$, respetive $\min$, na konečné množině 
strategií, pak nutně musí $\max$, respektive $\min$, existovat. \\ A to tedy budou optimální strategie. $\qed$

\subsection{Sedlový bod typu maxmin}\label{sedlmax}
Nechť $f: M \times N \rightarrow \R$. Řekněme, že $(\hat x, \hat y) \in M \times N$ je \textbf{sedlový bod} funkce f, 
jestliže pro každé $x \in M$ a každé $y \in N$ je 
\[ 
    f(x, \hat y) \leq f(\hat x, \hat y) \leq f(\hat x, y).
\]
\newpage
\subsection{Vztah Nashova equilibria a sedlového bodu}\label{sedl}
Ať $G = (S_1, S_2, u)$ je hra dvou hráčů s nulovým součtem a $(\hat \sigma, \hat \tau) \in S_1 \times S_2$. Potom 
$(\hat \sigma, \hat \tau)$ je \hyperref[nash]{Nashovo equilibrium} hry $G$ právě tehdy, když $(\hat \sigma, \hat \tau)$
je \hyperref[sedlmax]{sedlový bod} funkce $u$.

Důkaz.

$(\hat \sigma, \hat \tau)$ je N. e., tj. 
\begin{align*}
    u(\sigma, \hat \tau) &\leq u(\hat \sigma, \hat \tau) \quad \forall \sigma \in S_1 \\
    -u(\hat \sigma, \tau) &\leq -u(\hat \sigma, \hat \tau) \quad \forall \tau \in S_2 \\
    &\,\, \Big\Updownarrow \\
    u(\sigma, \hat \tau) \leq u (\hat \sigma&, \hat \tau) \leq u(\hat \sigma, \tau) \quad \forall \sigma \in S_1, 
    \tau \in S_2
\end{align*}
Což je přesně \hyperref[sedlmax]{sedlový bod} funkce $u$. $\qed$

\subsection{Věta o Nashově equilibriu a optimálních strategiích}\label{nashOpt}
Nechť $G = (S_1, S_2, u)$ je hra dvou hráčů s nulovým součtem.
\begin{enumerate}[(a)]
    \item Je-li $(\hat \sigma, \hat \tau) \in S_1 \times S_2$ \hyperref[nash]{Nashovo equilibrium} hry $G$, pak 
    $v = u(\hat \sigma, \hat \tau)$ je cena hry $G$, $\hat \sigma$ je \hyperref[optimalStrat]{optimální strategie} 
    prvního hráče a $\hat \tau$ je optimální strategie druhého hráče.
    \item Jestliže $v$ je cena hry $G$, $\hat \sigma$ je optimální strategie prvního hráče a $\hat \tau$ je 
    \hyperref[optimalStrat]{optimální strategie} druhého hráče, pak 
    $v = u(\hat \sigma, \hat \tau)$ a $(\hat \sigma, \hat \tau)$ je \hyperref[nash]{Nashovo equilibrium}.
\end{enumerate}
Důkaz (a).

$(\hat \sigma, \hat \tau)$ je N. e. $\implies$ 
$\underbrace{u(\sigma, \hat \tau) \leq u(\hat \sigma, \hat \tau) \leq u(\hat\sigma, \tau) \quad \forall \sigma \in S_1, 
\forall \tau \in S_2}_{(*)}$ $\dots$ N. e. je \hyperref[sedl]{sedlový bod} funkce $u$.
\[
    (*) \implies \overline{v} = \inf_{\tau \in S_2} \sup_{\sigma \in S_1} u(\sigma, \tau) \leq 
    \sup_{\sigma \in S_1} u(\sigma, \hat \tau) \leq u(\hat \sigma, \hat \tau) \leq 
    \inf_{\tau \in S_2} u(\hat\sigma, \tau) \leq
    \sup_{\sigma \in S_1} \inf_{\tau \in S_2} u(\sigma, \tau) = \underline{v}
\]
Tedy $\overline{v} \leq \underline{v}$. Navíc již víme, že $\underline{v} \leq \overline{v}$. 
Proto $\underline{v} = \overline{v}$.

Odtud $v = \underline{v} = \overline{v} = u(\hat \sigma, \hat \tau)$, $\hat \sigma$ je optimální strategie 1. hráče a 
$\hat \tau$ je optimální strategie 2. hráče. $\qed$

Důkaz (b).

Z předpokladu plyne: 
\[  
    \sup_{\sigma \in S_1} u(\sigma, \hat \tau) = v = \inf_{\tau \in S_2} u(\hat \sigma, \tau)
\]
Proto:
\[
    u(\sigma, \hat \tau) \leq v \leq u(\sigma, \hat \tau) \quad \forall \sigma \in S_1, \tau \in S_2
\]
A z toho nutně plyne $v = u(\hat \sigma, \hat \tau)$ když dosadíme $\sigma = \hat \sigma$ a $\tau = \hat \tau$. 

Tedy platí 
\[
    u(\sigma, \hat \tau) \leq u(\hat \sigma, \hat \tau) \leq u(\hat\sigma, \tau) \quad \forall \sigma \in S_1, 
    \forall \tau \in S_2 \implies (\hat \sigma, \hat \tau) \text{ je N. e. } \qed
\]
