\section{Třetí týden}

\subsection{Metoda nejmenších čtverců}

\begin{multicols}{2}
    \begin{tikzpicture}
        \draw[->] (-2,0) -- (2,0) node[right] {\(x\)};
        \draw[->] (0,-2) -- (0,2) node[above] {\(y\)};
        
        \draw[blue, thick] (-1.5,-1.5) -- (1.5,1.5) node[right] {\(L = \bc{Ax \mid x \in \R^n}\)};
    
        \filldraw[red, thick] (-1, 2) circle (2pt) node[above] {$b$};
        \filldraw[red, thick] (0.5, 0.5) circle (2pt) node[right] {$P_L (b)$};
        \draw[red, dashed] (-1, 2) -- (0.5, 0.5);
    \end{tikzpicture}

\columnbreak
    Pokud $b \in L$, řešíme úlohu $Ax = b$. \\Pokud $b \not\in L$, řešíme $Ax = P_L (b)$.
    \[
        \argmin_{x \in \R^n} \| Ax - b\| = \argmin_{x \in \R^n} \| Ax - b\|^2
    \]
\end{multicols}
Důkaz.

Chceme ukázat, že $\hat x \in \underset{x \in \R^n}{\argmin} \| Ax - b^2\| \iff A^T A \hat x = A^T b$.

\enquote{$\Rightarrow$}: Ať $A \hat x = P_L (b) \stackrel{\hyperref[varNer]{(2)}}{=} b - P_{L^\perp} (b) \quad / \cdot A^T$
\[
    A^T A \hat x = A^T  b - \underbrace{A^T P_{L^\perp} (b)}_{\stackrel{?}{=0}}
\]
\[
    \rightarrow \| A^T P_{L^\perp} (b)\|^2 = \langle A^T P_{L^\perp} (b), A^T P_{L^\perp} (b)\rangle = 
    \langle \underbrace{P_{L^\perp} (b)}_{\in L^\perp}, \underbrace{(A^T)^T (A^T P_{L^\perp} (b))}_{\in L}\rangle = 0. \qed
\]

\enquote{$\Leftarrow$}: Ať $A^T A \hat x = A^T b$.\\
Ať $x \in \R^n$.
\[
    0 = \langle \underbrace{x, A^T A \hat x - A^T b}_{A^T (A \hat x - b)}\rangle = 
    \langle \underbrace{(A^T)^T x}_L, A \hat x - b\rangle \implies A \hat x - b \in L^\perp
\]
\[
    \rightarrow b = \underbrace{A \hat x}_{\in L} + \underbrace{(b-A \hat x)}_{L^\perp} 
    \stackrel{\hyperref[ortoRoz]{(c)}}{\implies} A \hat x = P_L (b). \qed
\]

\subsection{Příklad výpočtu metody nejmenších čtverců}
$
A=
\begin{bmatrix}
    1 & 0 \\
    0 & 1 \\
    1 & 1
\end{bmatrix},
b=
\begin{bmatrix}
    1 \\
    1 \\
    1 
\end{bmatrix}$.

$A^T A \hat x = A^T b$

$A^T A = 
\begin{bmatrix}
    1 & 0 & 1 \\
    0 & 1 & 1
\end{bmatrix}
\begin{bmatrix}
    1 & 0 \\
    0 & 1 \\
    1 & 1
\end{bmatrix}
=
\begin{bmatrix}
    2 & 1 \\
    1 & 2 
\end{bmatrix} 
\rightarrow \det = 3 \implies \text{existuje inverze.}$
\\

$(A^T A)^{-1} = \frac{1}{3}
\begin{bmatrix}
    2 & -1 \\
    -1 & 2 
\end{bmatrix} \implies \hat x = (A^T A)^{-1}A^T b = \frac{1}{3}
\begin{bmatrix}
    2 & -1 \\
    -1 & 2 
\end{bmatrix}
\begin{bmatrix}
    1 & 0 & 1 \\
    0 & 1 & 1
\end{bmatrix}
\begin{bmatrix}
    1 \\
    1 \\
    1 
\end{bmatrix}\\ 
= \frac{1}{3}
\begin{bmatrix}
    2 & -1 \\
    -1 & 2 
\end{bmatrix}
\begin{bmatrix}
    2 \\
    2 
\end{bmatrix} = \frac{1}{3}
\begin{bmatrix}
    2 \\
    2 
\end{bmatrix}$.

\subsection{Příklad výpočtu metody nejmenších čtverců}
V rovině jsou dány body $(0, -\frac{1}{2})^T, (1, \frac{1}{3})^T$ a $(2, \frac{2}{3})^T$. Pomocí metody nejmenších 
čtverců proložme těmito body přímku o rovnici $y = kx + q$, kde $k ,q \in \R$.

\[
\begin{rcases*}
    0k + q = -\frac{1}{2} \\
    1k + q = \frac{1}{3} \\
    2k + q = \frac{2}{3}
\end{rcases*} 
A = 
\begin{bmatrix}
    0 & 1 \\
    1 & 1 \\
    2 & 1
\end{bmatrix},
b = 
\begingroup
    \renewcommand*{\arraystretch}{1.5}
    \begin{bmatrix}
        -\frac{1}{2} \\
        \phantom{-}\frac{1}{3} \\
        \phantom{-}\frac{2}{3}
    \end{bmatrix}
\endgroup
\]

$A^T A = 
\begin{bmatrix}
    0 & 1 & 2 \\
    1 & 1 & 1
\end{bmatrix}
\begin{bmatrix}
    0 & 1 \\
    1 & 1 \\
    2 & 1 
\end{bmatrix} =
\begin{bmatrix}
    5 & 3 \\
    3 & 3
\end{bmatrix}$

$(A^T A)^{-1} = \frac{1}{6}
\begin{bmatrix}
    \phantom{-}3 & -3 \\
    -3 & \phantom{-}5
\end{bmatrix}$

\[
    \hat x = \frac{1}{6}
    \begin{bmatrix}
        \phantom{-}3 & -3 \\
        -3 & \phantom{-}5
    \end{bmatrix}
    \begin{bmatrix}
        0 & 1 & 2 \\
        1 & 1 & 1
    \end{bmatrix}
    \begingroup
        \renewcommand*{\arraystretch}{1.5}
        \begin{bmatrix}
            -\frac{1}{2} \\
            \phantom{-}\frac{1}{3} \\
            \phantom{-}\frac{2}{3}
        \end{bmatrix}
    \endgroup = \frac{1}{6}
    \begin{bmatrix}
        \phantom{-}3 & -3 \\
        -3 & \phantom{-}5
    \end{bmatrix}
    \begingroup
        \renewcommand*{\arraystretch}{1.5}
        \begin{bmatrix}
            \frac{5}{3} \\
            \frac{1}{2}
        \end{bmatrix}
    \endgroup = \frac{1}{6}
    \begingroup
        \renewcommand*{\arraystretch}{1.5}
        \begin{bmatrix}
            \phantom{-}\frac{7}{2} \\
            -\frac{5}{2}
        \end{bmatrix}
    \endgroup = \frac{1}{12}
    \begin{bmatrix}
        \phantom{-}7 \\
        -5
    \end{bmatrix}.
\]

\subsection{Věta o oddělitelnosti bodu a konvexní množiny}
