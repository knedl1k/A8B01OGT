\section{Druhý týden}

\subsection{Věta o nejlepší aproximaci}
Je-li $C \subseteq \R^n$ neprázdná uzavřená konvexní množina, pak pro každé $x \in \R^n$ existuje právě jeden bod 
$\hat y \in C$ tak, že $\dist (x; C) = ||x-\hat y||$.

Důkaz.

1. Existence\\
Cíl: Existuje bod minima\\
Úvaha:
\begin{multicols}{2}

    \begin{tikzpicture}[fill=gray]
        \begin{scope}
            \clip (0,0) circle (2);
            \clip (3,0) circle (2);
            \fill[pattern=north east lines] (0,0) circle (2);
        \end{scope}
        
        \draw (0,0) circle (2) node [text=black] {$M$};;
        \draw (3,0) circle (2);
    
        \filldraw[black] (1, 0) circle (2pt) node[left] {$z$};
        \filldraw[black] (2, 0) circle (2pt) node[right] {$y$};
        \filldraw[black] (3, 0) circle (2pt) node[right] {$x$};
    
        \draw[->, thick] (2,-2) -- (1.8,-0.5) node[midway, above, sloped] {$C_z \not= \emptyset$}; % TODO upravit sipku
    \end{tikzpicture}

\columnbreak
    $M$ je obecná konvexní množina.\\c x
    $R = ||x-z||$,\\
    $Cz = M \cap B (x, R) = M \cap \bc{a \in \R^n \mid ||z-a|| \leq R}$.\\
    $\uparrow$\\
    $\underbrace{\text{uzavřená, omezená}}_\text{kompakt}$, neprázdná

    Tedy $a \mapsto ||x-a||$ je spojitá.

    $\Rightarrow$ Spojitost na kompaktní mnoižině znamená, že $f$ nabývá na $C_z$ minima dle Weierstrassova kritéria.
\end{multicols}
Ať $y$ je bod minima. Všechny body v $M$ mají od $x$ vzdálenost $\geq ||x-y||. \qed$
\\

2. Jednoznačnost.\\
Cíl: Pokud $a,b \in \R^n : ||x-a|| = ||x-b|| = \overbrace{\dist(x,M)}^\delta$, pak $a=b$.\\
Lemma, rovnoběžníkové pravidlo: $u,v \in \R^n \Rightarrow ||u+v||^2 + ||u-v||^2 = 2\left(||u||^2 + ||v||^2\right)$.\\
Důkaz lemma: 
\[
    ||u+v||^2 + ||u-v||^2 = \langle u+v, u+v \rangle + \langle u-v, u-v \rangle = ||u||^2 + 2 \langle u, v \rangle + 
    ||v||^2 + ||u||^2 - 2 \langle u, v \rangle + ||v||^2
\] 
\[
    = 2\left( ||u||^2 + ||v||^2 \right) \text{.} \qed
\]
Důkaz jednoznačnosti:\\
Ať $y = \frac{1}{2}a + \frac{1}{2}b$. \\
Pak $\delta^2 \leq ||x-y||^2 = ||x-\frac{1}{2}a - \frac{1}{2}b||^2 = ||\frac{1}{2}(x-a) + \frac{1}{2}(x-b)||^2 = 
\frac{1}{4}||\underbrace{(x-a)}_{u} + \underbrace{(x-b)}_{v}||^2 \\ 
\stackrel{lemma}{=} \frac{1}{4} \left[ 2 \left(\underbrace{||x-a||^2}_{\delta^2} + \underbrace{||x-b||^2}_{\delta^2}\right) 
- \underbrace{||(x-a) + (x-b)||^2}_{b-a}\right] = \delta^2 - \frac{1}{4} ||b-a||^2 \Rightarrow \delta^2 \leq \delta^2 - 
\underbrace{\frac{1}{4} ||b-a||^2}_{\leq 0 \Rightarrow a=b} \text{.}$