\section{Konvexní množiny} \label{sec:konvex}
Definice. Množina $C \subseteq \R^n$ se nazve konvexní, jestliže pro každé $x, y \in C$ je $[x,y] \in C$.

\subsection{Uzavřená úsečka}
Nechť $x, y \in \R^n$. Množina
\[ [x,y] \coloneq \bc{\lambda x + (1-\lambda)y \mid 0 \leq \lambda \leq 1} \]
se nazývá uzavřená úsečka s krajními body $x$ a $y$.

\subsection{Je nadrovina konvexní?}
Definice nadroviny: $H(y; \alpha) \coloneq \bc{x \in \R^n \mid \langle x, y \rangle = \alpha}$, $y \in \R^n$, 
$\alpha \in \R$.

Důkaz.

Ať $x,z \in H(y; \alpha), \lambda \in [0,1]$.\\
Cíl: $\lambda x + (1-\lambda) z \in H(y; \alpha)$. Tedy dokazujeme podle \hyperref[sec:konvex]{definice}.

$\langle \lambda x + (1-\lambda)z, y \rangle = \lambda \underbrace{\langle x,y \rangle}_{\alpha} + (1-\lambda)
\underbrace{\langle z,y \rangle}_{\alpha} = \lambda \alpha + (1-\lambda) \alpha = \alpha$.

$\Rightarrow \lambda x + (1-\lambda)z \in H(y; \alpha). \qed$

\subsection{Je uzavřený poloprostor konvexní?}
Nechť $y \in \R^n \setminus \bc{0}$ a $\alpha \in \R$. \\ Uzavřený poloprostor 
$P(y; \alpha) = \bc{x \in \R^n \mid \langle x, y\rangle \leq \alpha}$ je konvexní množina.

Důkaz.

Ať $a,b \in P(y; \alpha), \lambda \in [0,1]$.\\
Cíl: $\lambda a + (1-\lambda) b \in P(y, \alpha)$. Tedy dokazujeme podle \hyperref[sec:konvex]{definice}.

$\langle \lambda a + (1-\lambda)b, y\rangle = \lambda \underbrace{\langle a, y\rangle}_{\leq \alpha} + 
(1-\lambda)\underbrace{\langle b, y\rangle}_{\leq \alpha} \leq \alpha$

$\Rightarrow \lambda a + (1-\lambda)b \in P(y; \alpha). \qed$

\subsection{Je uzavřená koule konvexní?}
Definice uzavřené koule: $B(a; r) = \bc{a \in \R^n \mid \|x -a \| \leq r}$, o středu $a \in \R^n$ a poloměru $r > 0$.

Důkaz.

Ať $x,y \in \R^n, \lambda \in [0,1]$.\\
Cíl: $\| [\lambda x + (1-\alpha)y] - a \| \leq r$. Tedy za $x$ z definice dosadíme úsečku mezi body $x$ a $y$, které 
jsme si vybrali a chceme ukázat, že i tato úsečka leží v uzavřené kouli, dle \hyperref[sec:konvex]{definice}.

\[
    \| [\lambda x + (1-\alpha)y] - a \| = \| \lambda x - (1-\lambda)a + (1-\lambda)y - \lambda a \| =
    \| \lambda (x-a) + (1-\lambda)(y-a) \|
\]
\[
    \leq \lambda \|\underbrace{x-a}_{\leq r}\| +  (1-\lambda)\|\underbrace{y-a}_{\leq r}\| \leq \lambda r + (1-\lambda)r
     = r. \qed
\]

\subsection{Je okolí konvexní?}
Definice okolí: $B(a;r) = \bc{a \in \R^n \mid \|x -a \| < r}$, o středu $a \in \R^n$ a poloměru $r > 0$.

Důkaz.

Ať $x,y \in \R^n, \lambda \in [0,1]$.\\
Cíl: $\| [\lambda x + (1-\alpha)y] - a \| < r$. Dle \hyperref[sec:konvex]{definice}.

\[
    \| [\lambda x + (1-\alpha)y] - a \| = \| \lambda x - (1-\lambda)a + (1-\lambda)y - \lambda a \| =
    \| \lambda (x-a) + (1-\lambda)(y-a) \|
\]
\[
    \leq \lambda \|\underbrace{x-a}_{< r}\| +  (1-\lambda)\|\underbrace{y-a}_{< r}\| < \lambda r + (1-\lambda)r
     = r. \qed
\]

\subsection{Je průnik množin konvexní?}
Úvaha pro 2 množiny ve $\R^2$:

\begin{wrapfigure}{r}{0.25\textwidth}
    \vspace{-6em}
    \hspace*{-2em}
    \begin{tikzpicture}[remember picture]
        % y > 0 ("/")
        \fill[pattern=north east lines, pattern color=blue, opacity=0.6] (-2,0) rectangle (2,2);

        % x > 0 ("\")
        \fill[pattern=north west lines, pattern color=red, opacity=0.6] (0,-2) rectangle (2,2);

        \draw[thick,->] (-2.2,0) -- (2.2,0) node[anchor=north] {\(x\)};
        \draw[thick,->] (0,-2.2) -- (0,2.2) node[anchor=east] {\(y\)};
        \node[anchor=north east] at (0,0) {\(0\)};
    \end{tikzpicture}
\end{wrapfigure}

Mějme jednu modrou ($y \geq 0$) a druhou červenou ($x \geq 0$) \hyperref[sec:konvex]{konvexní} \\ množinu. Jejich průnik 
je pak nezáporný ortant, tedy \\
$\R_+^n = \bc{(x_1, \dots, x_n)^T \in \R^n \mid x_1 \geq 0, \dots, x_n \geq 0}$.

Visuálně je průnik nekonvexní.

Důkaz.

Nechť ${x,y \in \bigcap\limits_{i \in I} \M_{i}, \forall i \in I \implies [x, y] \in \M_i, \forall i \in I
\implies [x,y] \subseteq \bigcap\limits_{i \in I} \M_{i}.}$

\subsection{Důkaz, že rozdíl a sjednocení nezachovává konvexitu}
Mějme $[0,1] \setminus (0,1) = \bc{0,1} = \bc{0} \cup \bc{1}$.

$[0,1]$ a $(0,1)$ jsou \hyperref[sec:konvex]{konvexní} množiny. Jejich rozdíl ale už konvexní není.\\
$\bc{0}$ a $\bc{1}$ jsou konvexní množiny. Jejich sjednocení ale už konvexní není.

\section*{Afinní zobrazení} \label{sec:afin}
Definice. Zobrazení $f: \R^n \rightarrow \R^m$ se nazývá afinní, existují-li $A \in \M_{m,n} (\R)$ a $b \in \R^m$
tak, že \[f(x) = Ax + b.\]

\subsection{Důkaz, že afinní zobrazení je konvexní}
Tvrzení.

Nechť $f: \R^n \rightarrow \R^m$. Pak $f$ je \hyperref[sec:afin]{afinní} $\iff$ pro každé $x,y \in \R^n$ a každé
$\lambda \in \R$ platí
\[f(\lambda x + (1-\lambda) y) =\lambda f(x) + (1-\lambda) f(y)\text{.}\]

Důkaz.

\enquote{$\Rightarrow$}: Ať $f(x) = Ax + b$, kde $A \in \M_{m,n} (\R)$, $b \in \R^n$.

Ať $x, y \in \R^n, \lambda \in \R$.
\[
    f(\lambda x + (1 - \lambda) y) = A [\lambda x + (1-\lambda) y] + b = \lambda A x + (1-\lambda)Ay + \lambda b +
    (1-\lambda)b =
\]
\[
    \lambda \underbrace{(Ax + b)}_{f(x)} + (1-\lambda)\underbrace{(Ay + b)}_{f(y)} = \lambda f(x) + (1-\lambda)f(y). 
    \qed
\]

\enquote{$\Leftarrow$}:
Cíl: Ukázat, že $f$ je \hyperref[sec:afin]{afinní}, tedy $f(x) = Ax + b$.\\
Zvolme $\varphi(x) = f(x) - f(0)$.\\
Pokud je $f$ \hyperref[sec:afin]{afinní}, pak zobrazení $\varphi$ by mělo být dáno jako $Ax$, tedy být lineární.\\
Cíl: $\varphi$ je lineární zobrazení.

Musíme ověřit uzavřenost na násobení a sčítání z definice.

(1) Ať $x \in \R^n$, $\alpha \in R$.\\
Cíl: $\varphi(\alpha x) = \alpha \varphi(x)$.
\[
    \varphi(\alpha x) = f(\alpha x) - f(0) = f(\alpha x + (1-\alpha)0) - f(0) = \alpha f(x) + (1-\alpha)f(0) - f(0) =
\]
\[
    \alpha f(x) - \alpha f(0) = \alpha (f(x) - f(0)) = \alpha \varphi (x-0). \qed
\]

(2) Ať $x, y \in \R^n$.\\
Cíl: $\varphi(x+y) = \varphi(x) + \varphi(y)$.
\[
    \varphi(x+y) = 
    \varphi \left(2 \left(\frac{1}{2} (x+y)\right)\right) \stackrel{(1)}{=} 2 \varphi \left(\frac{1}{2} (x+y)\right) =
    2 \left[f(\frac{1}{2}x + \frac{1}{2}y) - f(0)\right] = 
\]
\[
    2 \left[\frac{1}{2} f(x) + \frac{1}{2}f(y) - f(0) \right] = f(x) + f(y) - f(0) - f(0) = 
    \underbrace{f(x) - f(0)}_{\varphi(x)} + \underbrace{f(y) - f(0)}_{\varphi(y)} =
    \varphi(x) + \varphi(y). \qed
\]

\subsection{Důkaz, že obraz konvexní množiny při afinním zobrazení je konvexní}
Tvrzení.

Je-li $f: \R^n \rightarrow \R^m$ \hyperref[sec:afin]{afinní} a $C \subseteq \R^n$ \hyperref[sec:konvex]{konvexní}, pak
$f(C)$ je konvexní.

Důkaz.

Mějme $a, b \in f(C) \implies \exists x, y \in C: f(x)=a, f(y)=b$.

Dle předpokladu je $C$ konvexní. $\implies [x, y] \subseteq C \implies \underbrace{f([x, y])}_{\subseteq f(C)} =
[\underbrace{f(x)}_a, \underbrace{f(y)}_b] \subseteq f(C). \qed$

\subsection{Důkaz, že kartézský součin je konvexní}
Tvrzení.

Nechť $C_1 \subseteq \R^n$ a $C_2 \subseteq \R^m$. Pak $C_1$ a $C_2$ jsou \hyperref[sec:konvex]{konvexní} množiny právě 
tehdy, když $C_1 \bigtimes C_2$ je konvexní množina.

Důkaz.

\enquote{$\Rightarrow$}: Mějme
$
\begin{bmatrix}
    a \\
    b
\end{bmatrix}\hspace{-1mm}\text{,}
\begin{bmatrix}
    c \\
    d
\end{bmatrix} \in C_1 \bigtimes C_2, \lambda \in [0,1]$\\
Cíl:
$
\lambda \begin{bmatrix}
    a \\
    b
\end{bmatrix}
+ (1-\lambda)
\begin{bmatrix}
    c \\
    d
\end{bmatrix} \in C_1 \bigtimes C_2.$ Dle \hyperref[sec:konvex]{definice}.

$
\lambda \begin{bmatrix}
    a \\
    b
\end{bmatrix}
+ (1-\lambda)
\begin{bmatrix}
    c \\
    d
\end{bmatrix} =
\begin{bmatrix}
    \lambda a \\
    \lambda b
\end{bmatrix}
+
\begin{bmatrix}
    (1-\lambda)c \\
    (1-\lambda)d
\end{bmatrix}
=
\begin{bmatrix} % TODO: add overbrace?
    \lambda a + (1-\lambda)c \\
    \lambda b + (1-\lambda)d
\end{bmatrix} \in C_1 \bigtimes C_2. \qed$

\enquote{$\Leftarrow$}: Definujme \hyperref[sec:afin]{afinní} zobrazení $f : \R^n \times \R^m \rightarrow \R^n$ 
předpisem \[ f(x,y) = x \text{.}\]
Pak $f$ je afinní. Navíc $f(C_1 \bigtimes C_2) = C_1$. $\implies C_1$ je \hyperref[sec:konvex]{konvexní}, protože afinní
zobrazení zachovává konvexitu.
A důkaz bude obdobný pro $C_2$, zde zadefinujme afinní zobr. $g : \R^n \times \R^m \rightarrow \R^n$ předpisem
\[ g(x,y) = y \text{.}\]
Pak $g$ je afinní. Navíc $g(C_1 \bigtimes C_2) = C_2$. $\implies C_2$ je konvexní, protože afinní zobrazení zachovává
konvexitu. $\qed$

% \subsection{Práce s ortonormální bází a skalárním součinem}
% Uvažme lineární prostor $\S^n = \bc{A \in \M_n (\R) \mid A^ T = T}$ reálných symetrických $n \times n$
% matic se skalárním součinem $\langle A B \rangle _{\S_n} = Tr (AB)$.
% \begin{enumerate}[(a)]
%     \item Ukažte, že $
%     \begin{bmatrix}
%         \begin{pmatrix}
%             1 & 0 \\
%             0 & 0
%         \end{pmatrix},
%         & \frac{1}{\sqrt{2}}
%         \begin{pmatrix}
%             0 & 1 \\
%             1 & 0
%         \end{pmatrix},
%         &
%         \begin{pmatrix}
%             0 & 0 \\
%             0 & 1
%         \end{pmatrix}
%     \end{bmatrix}$
%     je ortonormální báze na $\S^2$.
%     \item Ukažte, že zobrazení
%     \[ \varphi:
%     \begin{bmatrix}
%         a & b \\
%         b & c
%     \end{bmatrix} \in \S^2 \mapsto
%     \begin{bmatrix}
%         a \\
%         \sqrt{2}b \\
%         c
%     \end{bmatrix} \in \R^3
%     \]
%     je isomorfismus lineárního prostoru $\S^2$ na $\R^3$ zachovávající skalární součin (tj. \\ $\langle A, B \rangle _{\S_2} = \langle \varphi(A), \varphi(B) \rangle$ pro všechna $A,B \in
%     \S^2$, kde $\langle \dots \rangle$ je standardní skalární součin na $\R^3$)
%     \item Zobecněte výsledky bodů (a) a (b) do prostoru $\S^2 na \R^2$ zachovávající skalární součin.
%     \item Ať $\S^2_+$ je množina všech reálných symetrických $2 \times 2$ matic, které jsou navíc positivně
%     semidefinitní. Ukažte, že jestliže $\varphi$ je zobrazení z bodu (b), pak
%     \[\varphi(\S^2_+) = \bc{(x,y,z)^T \in \R^3 \mid x \geq 0, z \geq 0, 2xz - y^2 \geq 0} \text{.}\]
% \end{enumerate}

% \subsection{Bijekce mezi dvěma optimalisačními úlohami}
% Je dána úloha
% \begin{align*}
%     \text{minimalisujte } & \langle X, A \rangle _{\S_2} \\
%     \text{za podmínek }   & \langle X, \bb{1} \rangle _{\S_2} = 2, \\
%                           & X \in \S_+^2,
% \end{align*}
% kde $A =
%     \begin{bmatrix}
%         3 & 1 \\
%         1 & 1
%     \end{bmatrix}
% $ a $\bb{1} =
%     \begin{bmatrix}
%         1 & 0 \\
%         0 & 1
%     \end{bmatrix}
% $. Ukažte\footnote{Nápověda: využijte výsledků 7. a 8. příkladu.}, že existuje bijekce mezi množinou všech jejich řešení a množinou všech řešení úlohy
% \begin{align*}
%     \text{minimalisujte } & 3x_1 + 2x_2 + x_3 \\
%     \text{za podmínek }   & x_1 + x_3 = 2, \\
%                           & x_1 x_3 - x_2^2 \geq 0,
%                           & x_1, x_3 \geq 0.
% \end{align*}

% \subsection{Bijekce mezi dvěma optimalisačními úlohami}
% Je dána úloha
% \begin{align*}
%     \text{minimalisujte } & \langle X, A \rangle _{\S_2} \\
%     \text{za podmínek }   & \langle X, B \rangle _{\S_2} = 0, \\
%                           & \langle X, \bb{1} \rangle _{\S_2} = 1, \\
%                           & X \in \S_+^2,
% \end{align*}
% kde $A =
%     \begin{bmatrix}
%         2 & 0 \\
%         0 & -1
%     \end{bmatrix}
% $, $B =
%     \begin{bmatrix}
%         0 & 1 \\
%         1 & 0
%     \end{bmatrix}
% $ a $\bb{1} =
%     \begin{bmatrix}
%         1 & 0 \\
%         0 & 1
%     \end{bmatrix}
% $. Ukažte\footnote{Nápověda: využijte výsledků 7. a 8. příkladu.}, že existuje bijekce mezi množinou všech jejich řešení a množinou všech řešení úlohy
% \begin{align*}
%     \text{minimalisujte } & 2x-y \\
%     \text{za podmínek }   & x+y=1, \\
%                           & x, y \geq 0.
% \end{align*}

\subsection{Určení definitnosti matic}
Určete definitnost matice $A$, jestliže
\begin{enumerate}[(a)]
    \item
    $\begin{bmatrix}
        9 & 6 \\
        6 & 4
    \end{bmatrix}$;
    \item
    $\begin{bmatrix}
        15 & 3 & 2 \\
        3 & 1 & 0 \\
        2 & 0 & 1
    \end{bmatrix}$;
    \item
    $\begin{bmatrix}
        4 & 2 & 2 \\
        2 & 1 & 1 \\
        2 & 1 & 0
    \end{bmatrix}$;
    \item
    $\begin{bmatrix}
        3 & 2 & 1 \\
        2 & 1 & 1 \\
        1 & 1 & 0
    \end{bmatrix}$;
    \item
    $\begin{bmatrix}
        -1 & \phantom{-}0 & \phantom{-}1 \\
        \phantom{-}0 & -2 & \phantom{-}2 \\
        \phantom{-}1 & \phantom{-}2 & -3
    \end{bmatrix}$;
    \item
    $\begin{bmatrix}
        1 & 2 & 0 \\
        2 & 5 & 1 \\
        0 & 1 & 1
    \end{bmatrix}$.
\end{enumerate}

Matice, u které chceme určovat definitnost, musí být $\underbrace{\text{symetrická}}_{Q = Q^T}$.

Pak platí:
\begin{align*}
    \langle Qx , x\rangle & \geq 0 \forall x \in \R^n \iff Q \text{ je positivně semidefinitní.}\\
    \langle Qx , x\rangle & > 0 \forall x \in \R^n \iff Q \text{ je positivně definitní.}
\end{align*} 
Analogicky pro negativně semidefinitní, respektive definitní. \\ Matice je indefitní pokud nesplňuje ani jednu možnost. 

Pro symetrické matice také platí, že $Q$ je negativně (semi)defitní, jestliže $(-Q)$ je positivně (semi)defintní. 
\label{matiVlastnost}

Pomocí Sylvesterova kritéria lze určit positivní, či negativní definitnost. Pro případy podezření na semidefinitnost 
je potřeba navíc prozkoumat menší minory matice.

(a) 
$\begin{bmatrix}
    9 & 6 \\
    6 & 4
\end{bmatrix}
\rightarrow |9| = 9 > 0,
\begin{vmatrix}
    9 & 6 \\
    6 & 4
\end{vmatrix} = 36 - 36 = 0.
\rightarrow$ podezření na positivní semidefinitnost.

Hlavní minory jsou $Q_{\bc{1}}$ a $Q_{\bc{1,2}}$.\\
Menší minory: $Q_I$, kde $I \subseteq \bc{1, \dots, n} \text{neprázdná}$. Aby matice byla positivně semidefinitní, tak 
det$Q_I \geq 0$.

Tedy: $Q_{\bc{2}} = 
\begin{bmatrix}
    4
\end{bmatrix}$. det $Q_{\bc{2}} = 4 > 0$.

Tedy matice
$\begin{bmatrix}
    9 & 6 \\
    6 & 4
\end{bmatrix}$ je positivně semidefinitní.

(b) 
$\begin{vmatrix}
    15 & 3 & 2 \\
    3 & 1 & 0 \\
    2 & 0 & 1
\end{vmatrix}
\begin{array}{l}
    R_1 - 2 R_3 \\
    R_2 \\
    R_3
\end{array} = 
\begin{vmatrix}
    11 & 3 & 0 \\
    3 & 1 & 0 \\
    2 & 0 & 1
\end{vmatrix} = 
\begin{vmatrix}
    11 & 3 \\
    3 & 1 
\end{vmatrix} = 11-9 = 2 >0.$ Matice je positivně definitní. 

(c)
$Q = \begin{bmatrix}
    4 & 2 & 2 \\
    2 & 1 & 1 \\
    2 & 1 & 0
\end{bmatrix}$

Pozorování: Matice je lineárně závislá, tedy det$Q = 0$. \\
$Q_{\bc{1}} = 4 > 0$,\\
$Q_{\bc{2}} = 1 > 0$,\\
$Q_{\bc{3}} = 0 = 0$.\\
Tedy matice je jedině positivně semidefinitní, nebo indefinitní.

Spočtěme tedy vedlejší minor, například vynechejme 1. řádek a 1. sloupec: 

$\begin{vmatrix}
    1 & 1 \\
    1 & 0
\end{vmatrix} = -1 < 0$. Aby matice $Q$ byla positivně semidefinitní, musely by i všechny vedlejší minory být $\geq 0$. 
Protože jsme našli případ, kdy tomu tak není, matice $Q$ je indefinitní.

% (d) % TODO: vyřešit definitnost
% $\begin{bmatrix}
%         3 & 2 & 1 \\
%         2 & 1 & 1 \\
%         1 & 1 & 0
% \end{bmatrix}$

(e)
$\begin{bmatrix}
    -1 & \phantom{-}0 & \phantom{-}1 \\
    \phantom{-}0 & -2 & \phantom{-}2 \\
    \phantom{-}1 & \phantom{-}2 & -3
\end{bmatrix}$

Pozorování: matice může být negativně (semi)definitní, nebo indefinitní.

Využijme tedy \hyperref[matiVlastnost]{vlastnosti} symetrických matic a určeme definitnost pro matici $(-Q)$.

$-Q = \begin{bmatrix}
    \phantom{-}1 & \phantom{-}0 & -1 \\
    \phantom{-}0 & \phantom{-}2 & -2 \\
    -1 & -2 & \phantom{-}3
\end{bmatrix}$

det$(-Q) = \begin{vmatrix}
    \phantom{-}1 & \phantom{-}0 & -1 \\
    \phantom{-}0 & \phantom{-}2 & -2 \\
    -1 & -2 & \phantom{-}3
\end{vmatrix}
\begin{array}{l}
    R_1 \\
    R_2 \\
    R_3 + R_1 + R_2
\end{array} = 
\begin{vmatrix}
    1 & 0 & -1 \\
    0 & 2 & -2 \\
    0 & 0 & \phantom{-}0
\end{vmatrix} = 0.$

Tedy matice $(-Q)$ je positivně semidefinitní, nebo indefinitní.

$\begin{vmatrix}
    1 & 0 \\
    0 & 2
\end{vmatrix} = 2 \geq 0$.
$\begin{vmatrix}
    \phantom{-}2 & -2 \\
    -2 & \phantom{-}3
\end{vmatrix} = 2 \geq 0$.
$\begin{vmatrix}
    \phantom{-}1 & -1 \\
    -1 & \phantom{-}3
\end{vmatrix} = 2 \geq 0$.

$\implies (-Q)$ je positivně semidefinitní $\iff Q$ je negativně semidefinitní.

% (f) % TODO: vyřešit definitnost
% $\begin{bmatrix}
%     1 & 2 & 0 \\
%     2 & 5 & 1 \\
%     0 & 1 & 1
% \end{bmatrix}$

\subsection{Existence matice}
Ať $A \in \M_n (\R)$.
\begin{enumerate}[(a)]
    \item Ukažte, že $\langle Ax, y \rangle = \langle x, A^T y \rangle$ pro všechna $x, y \in \R^n$.
    \item Ukažte, že existují matice $B, C \in \M_n (\R)$ takové, že $B^T = B$, $C^T = -C$ a $A = B + C$. Jsou
    matice $B$ a $C$ určeny jednoznačně?
    \item Ukažte, že existuje symetrická matice $B \in \M_n (\R)$ taková, že $\langle Ax, x \rangle =
    \langle Bx, x \rangle$.
\end{enumerate}

Zadefinujme si vlastnost skalárního součinu: $\langle a, b\rangle = b^T a$, kde $b^T = 
\begin{pmatrix}
    b_1, \dots, b_n
\end{pmatrix}, a = 
\begin{pmatrix}
    a_1 \\
    \vdots \\
    a_n
\end{pmatrix}$. \label{skalVlastnost}

(a)Využijme zmíněné \hyperref[skalVlastnost]{vlastnosti}. \label{aExistence}

$\langle Ax, y\rangle = y^T Ax = \underbrace{y^T (A^T)^T}_{(A^T y)^T} x = (A^T y)^T x = \langle x, A^T y\rangle. \qed$

(b) Pozorování: Matice $B$ je symetrická a matice $C$ je antisymetrická.

Zvolme:
$\begin{rcases}
    &B = \frac{1}{2} (A + A^T) \\
    &C = \frac{1}{2} (A - A^T)
\end{rcases}
B+C = A.$

$C^T = \frac{1}{2} (A-A^T)^T = \frac{1}{2} (A^T - A) = -\frac{1}{2} (A-A^T) = -C. \checkmark$

$B^T = \frac{1}{2} (A+A^T)^T = \frac{1}{2} (A^T + A) = \frac{1}{2} (A+A^T) = B. \checkmark \qed$


(c) $\langle C x, x\rangle \stackrel{?}{=} 0$

\[
    \langle Cx, x\rangle \stackrel{\hyperref[aExistence]{(a)}}{=} \langle x, C^T x\rangle \stackrel{-C = C^T}{=} 
    - \langle x, Cx\rangle = - \langle Cx, x\rangle = 0.
\]
Matice $C$ tedy nijak nepřispívá do výsledku. Takže platí $\langle Ax, x\rangle = \langle Bx, x\rangle. \qed$

% \subsection{Gradient vektorového součinu}
% Nalezněte $\nabla f(x)$ a $\nabla^2 f(x)$, jestliže
% \begin{enumerate}[(a)]
%     \item $f(x) = \langle x,c \rangle$, kde $c \in \R^n$;
%     \item $f(x) = \langle Ax, x \rangle$, kde $A \in \M_n (\R)$. Určete také $\nabla f(x)$ a $\nabla^2 f (x)$ za
%     dodatečného předpokladu, že $A$ je symetrická matice.
% \end{enumerate}
