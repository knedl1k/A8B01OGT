\section*{Úvod}

\hspace{0.8cm} Tento text není psán jako učebnice, nýbrž jako soubor řešených příkladů, u kterých je vždy uveden celý 
korektní postup a případné moje poznámky, které často nebývají formální, a tedy by neměly být používány při oficálním 
řešení problémů, například při zkoušce. Jedná se pouze o pokus předat probíranou látku z různých úhlů pohledu, pokud by 
korektní matematický nebyl dostatečně výřečný.

\hspace{0.8cm} Velmi ocením, pokud čtenáři zašlou své podněty, úpravy anebo připomínky k textu. Budu rád za všechnu 
konstruktivní kritiku a nápady na změny. Dejte mi také prosím vědět, pokud v textu objevíte překlepy, chyby a jiné.

Errata a aktuální verse textu bude na stránce \url{https://github.com/knedl1k/A8B01OGT}.

\textbf{Poděkování.} Rád bych poděkoval docentu Martinu Bohatovi nejen za zadání, okolo kterých je postavena celá sbírka,
ale také za celý předmět Optimalizace a teorie her.

\hspace{0.8cm} Text je vysázen makrem \LaTeX{} Leslieho Lamporta s využitím balíků \texttt{hypperref} \\ 
Sebastiana Rahtze a Heiko Oberdiek. 

\subsection*{Stručné informace o textu}
Všechny růžové texty jsou zároveň hypertextové odkazy. Často jsou použity u  
\href{https://www.youtube.com/watch?list=PLQL6z4JeTTQmbViFsZoQnUC0FjFZrzfjz&v=da8ogC0dtwM}{přednáškových} 
příkladů, pomocí nichž lze vidět ukázkové řešení příkladu na přednášce.

U každého příkladu je pro ušetření místa a zpřehlednění sbírky řešení jednotlivých příkladů ihned pod zadáním.

